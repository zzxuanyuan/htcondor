%%%%%%%%%%%%%%%%%%%%%%%%%%%%%%%%%%%%%%%%%%%%%%%%%%%%%%%%%%%%%%%%%%%%%%
\subsection{\label{sec:PVM-Install}
Installing the PVM Contrib Module} 
%%%%%%%%%%%%%%%%%%%%%%%%%%%%%%%%%%%%%%%%%%%%%%%%%%%%%%%%%%%%%%%%%%%%%%

For complete documentation on using PVM in Condor, see the section
entitled ``Parallel Applications in Condor: Condor-PVM'' in the
version 6.1 manual.
This manual can be found at
\Url{http://www.cs.wisc.edu/condor/manual/v6.1}.

To install the PVM contrib module, all you have to do is download
to appropriate binary module for whatever platform(s) you plan to use
for Condor-PVM.
Once you have downloaded each module, uncompressed and untarred it, you
will be left with a directory that contains a \File{pvm.tar},
\File{README} and so on.
The \File{pvm.tar} acts much like the \File{release.tar} file for a
main release. 
It contains all the binaries and supporting files you would install in
your release directory:
\begin{verbatim}
        sbin/condor_pvmd
        sbin/condor_pvmgs
        sbin/condor_shadow.pvm
        sbin/condor_starter.pvm
\end{verbatim}

Since these files do not exist in a main release, you can safely untar
the \File{pvm.tar} directly into your release directory, and you're
done installing the PVM contrib module.
Again, see the 6.1 manual for instructions on how to use PVM in
Condor.

