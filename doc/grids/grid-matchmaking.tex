%%%%%%%%%%%%%%%%%%%%%%%%%%%%%%%%%%%%%%%%%%%%%%%%%%
\subsection{\label{sec:Grid-Matchmaking}Matchmaking in the Grid Universe}
%%%%%%%%%%%%%%%%%%%%%%%%%%%%%%%%%%%%%%%%%%%%%%%%%%
\index{matchmaking!on the Grid}
\index{grid computing!matchmaking}

In a simple usage, the grid universe allows users to specify a single
grid site as a destination for jobs.
This is sufficient when a user knows exactly which
grid site they wish to use,
or a higher-level resource broker
(such as the European Data Grid's resource broker)
has decided which grid site should be used.

When a user has a variety of grid sites to choose from,
Condor allows matchmaking of grid universe jobs,
to decide which grid resource a job should run on. 
Please note that this form of matchmaking is relatively new.
There are some rough edges as continual improvement occurs.

To facilitate Condor's matching of jobs with grid resources,
both the jobs and the grid resources need to be involved.
The job must provide all commands needed to make the
job work on a matched grid resource.
The grid resource must identify itself to Condor,
specifying all necessary attributes, such that Condor
can properly make matches.
The grid resource identification is accomplished by 
using \Condor{advertise} to form a ClassAd for the
grid resource, which is then used by Condor to make matches.

%%%%%%%%%%%%%%%%%%%%%%%%%%%%%%%%%%%%%%%%%%%%%%%%%%
\subsubsection{Job Submission}
%%%%%%%%%%%%%%%%%%%%%%%%%%%%%%%%%%%%%%%%%%%%%%%%%%

To submit a grid universe job intended for a single, specific
\SubmitCmd{gt2} resource,
the submit description file for the job explicitly specifies
the resource:

\footnotesize
\begin{verbatim}
globusscheduler = grid.example.com/jobmanager
\end{verbatim}
\normalsize

If there were multiple \SubmitCmd{gt2} resources that
might be matched to the job,
the submit description file changes:

\footnotesize
\begin{verbatim}
globusscheduler = $$(scheduler_location)
requirements    = TARGET.scheduler_location =!= UNDEFINED
\end{verbatim}
\normalsize

The \SubmitCmd{globusscheduler} command uses a substitution
macro.
The substitution macro defines the value 
(of \Attr{scheduler\_location}) using attributes
as specified by the matched grid resource.
The \SubmitCmd{requirements} command further restricts that
the job may only run on a machine (grid resource) that
defines \Attr{scheduler\_location}.
Note that this attribute name is invented for this example.
To make matchmaking work in this way,
both the job (as used here within the submit description file)
and the grid resource (in its created and advertised ClassAd)
must agree upon the name of the attribute.

As a more complex example,
consider a job that wants not only to run on a \SubmitCmd{gt2} resource,
but on one that has the Bamboozle software installed.
The complete submit description file might appear:

\footnotesize
\begin{verbatim}
universe        = grid
grid_type       = gt2
executable      = analyze_bamboozle_data
output          = aaa.$(Cluster).out
error           = aaa.$(Cluster).err
log             = aaa.log
globusscheduler = $$(scheduler_location)
requirements    = (TARGET.HaveBamboozle == True) && (TARGET.scheduler_location =!= UNDEFINED)
leave_in_queue  = jobstatus == 4
queue
\end{verbatim}
\normalsize

Any grid resource which has the
\Attr{HaveBamboozle} attribute defined as well as
set to \Expr{True} is further checked to have the
\Attr{scheduler\_location} attribute defined.
Where this occurs, a match may be made (from the
job's point of view).
A grid resource that has one of these attributes defined,
but not the other results in no match being made.

If the job may run on any grid resource,
not only those with
\SubmitCmd{grid\_type} \SubmitCmd{gt2},
then the submit description file for the job
uses the substitution macro within the 
\SubmitCmd{grid\_resource} command:

\footnotesize
\begin{verbatim}
universe      = grid
grid_resource = $$(resource_type)
executable    = aaa
output        = aaa.$(Cluster).out
error         = aaa.$(Cluster).err
log           = aaa.log
requirements  = TARGET.resource_type =!= UNDEFINED
queue
\end{verbatim}
\normalsize

Any grid resource that advertises the attribute \Attr{resource\_type}
may be matched with the job.
At match time,
the substitution macro causes the
value given by \SubmitCmd{grid\_resource}
to specify the \SubmitCmd{grid\_type},
as well as set all other necessary grid-related job attributes.

%%%%%%%%%%%%%%%%%%%%%%%%%%%%%%%%%%%%%%%%%%%%%%%%%%
\subsubsection{Advertising Grid Resources to Condor}
%%%%%%%%%%%%%%%%%%%%%%%%%%%%%%%%%%%%%%%%%%%%%%%%%%

Any grid resource that wishes to be matched by Condor with
a job must advertise itself to Condor using a ClassAd.
To properly advertise, a ClassAd is sent
periodically to the \Condor{collector} daemon.
A ClassAd is a list of pairs, consisting of
attributes and values that describe an entity.
There are two entities relevant to Condor:
a job, and a machine.
A grid resource is a machine.
The ClassAd describes the grid resource, as well
as identifying the capabilities of grid resource.
It may also state both requirements and preferences
(called \SubmitCmd{rank}) for the jobs it will run.
See
Section~\ref{sec:matchmaking-with-classads} for an overview
of the interaction between matchmaking and ClassAds.
A list of common machine ClassAd attributes is given in
Section~\ref{user-man-machad}.

To advertise a grid site, place the attributes
in a file.
Here is a sample ClassAd that describes a grid resource
that is willing to run a
\SubmitCmd{grid\_type} \SubmitCmd{gt2} job:

\footnotesize
\begin{verbatim}
# example grid resource ClassAd for a gt2 job
MyType         = "Machine"
TargetType     = "Job"
Name           = "Example1_Gatekeeper"
Machine        = "Example1_Gatekeeper"
gatekeeper_url = "grid.example.com/jobmanager"
Requirements   = (CurMatches < 10) && (TARGET.JobUniverse == 9) && (TARGET.JobGridType =?= "gt2")
Rank           = 0.000000
CurrentRank    = 0.000000
WantAdRevaluate = True
UpdateSequenceNumber  = 4
CurMatches     = 0
\end{verbatim}
\normalsize

%% JobUniverse = 10 is the grid universe

\begin{verbatim}
MyType                = "Machine"
\end{verbatim}

Your grid site is pretending to be a single machine, for the purpose
of matchmaking. \Attr{MyType} is an attribute that the \Condor{negotiator}
daemon
will expect to be a string. Strings must be surrounded by double-quote
marks, as in this example. You may have surprising, unintuitive errors
if they are not quoted. You will always want \Attr{MyType} to be
``Machine''. 

\begin{verbatim}
TargetType            = "Job"
\end{verbatim}

This is an attribute that says the grid site (machine) wants to be
matched with a job. Leave this as it is. 


\footnotesize
\begin{verbatim}
Name                  = "Example1_Gatekeeper"
\end{verbatim}
\normalsize

You will want a unique name for each grid site. Any name is fine, as long as
it is quoted.

\footnotesize
\begin{verbatim}
Machine               = "Example1_Gatekeeper"
\end{verbatim}
\normalsize

Machine is just like Name. Normally in Condor, the Machine and Name
may be slightly different if you have multiple CPUs. For grid
matchmaking, they should probably be the same.

\footnotesize
\begin{verbatim}
gatekeeper_url        = "grid.example.com/jobmanager"
\end{verbatim}
\normalsize

This is the Globus gatekeeper contact string for your grid site. It is
probably a machine name followed by a slash followed by the name of
the jobmanager. If you have different job managers, you can only
specify one per ClassAd. 

\begin{verbatim}
UpdateSequenceNumber  = 4
\end{verbatim}

UpdateSequenceNumber is a positive number that must increase each time
you advertise a grid site. Normally you advertise your grid site
every five minutes. The \Condor{collector} daemon will discard a grid site's
ClassAd after 15 minutes if there have been no updates. A good number
to set this to is the current time in seconds (the epoch, as given by
the C \Procedure{time} function call), but if you are worried about your clock
running backward, you can set it to whatever you like. If ClassAds are
received with a sequence number older than the last ClassAd, they are
ignored. 

\begin{verbatim}
CurMatches            = 0
\end{verbatim}

This number is incremented each time a match is made for this grid
site. Unlike a normal machine ClassAd that can only be matched against
once, grid site advertisements can be matched against many time. 

You will probably want to set this number to be the number of grid
jobs that you have running on your site, and keep it updated each time
you submit a new ClassAd. If you do not specify CurMatches, Condor
will assume it is 0.

Condor will increment this number every time it makes a match against
a grid site.

\footnotesize
\begin{verbatim}
Requirements          = (CurMatches < 10) && (TARGET.JobUniverse == 9) && (TARGET.JobGridType =?= ``gt2'')
\end{verbatim}
\normalsize

These are the requirements that the grid site insists must be true
before it will accept a job. These could refer to features of the
job's ClassAd. In this case, we will take any grid universe job that's
of grid-type gt2, as
long we have less than 10 matches currently. This will ensure that
Condor will only run 10 jobs at your site---assuming that you keep
CurMatches up to date when jobs finish. Of course, you can edit this
statement to have different requirements. For example, if you want to
accept all jobs, you can have \ShortExpr{Requirements = True}.

\footnotesize
\begin{verbatim}
Rank                  = 0.000000
CurrentRank           = 0.000000
\end{verbatim}
\normalsize

This is a numerical ranking that will be assigned to a job. Right now
it is not used, but should be set to 0. 

\begin{verbatim}
WantAdRevaluate       = True
\end{verbatim}

The \Attr{WantAdRevaluate} attribute distinguishes grid site
ClassAds from normal machine ClassAds and allows multiple matches to
be made against a single site. It should be in your ad and should be
true. Note that True is not in quotes, and it should not be.

You can add other attributes to your ClassAd, to make it easy for a
job to decide which grid site it wants to use. For instance, if you
have pre-installed the Bamboozle software environment on your grid
site, you could advertise, \ShortExpr{HaveBamboozle = True} and
\ShortExpr{BamboozleVersion = 10}. Jobs can require a grid site that has
Bamboozle installed by extending their requirements with
\ShortExpr{HaveBamboozle == True}. (Note the double equal sign in the
requirements.) 

As an aside, we recommend that jobs that need specific applications
should bring them with them instead of relying on having them
pre-installed at a Grid site. You will have more reliable execution if
you do. 

Once you have a file that describes your site, you need to send it to
the \Condor{collector} daemon. For this, use \Condor{advertise}.
We recommend that you write a script to create the file
containing the ClassAd, then run the script every five minutes with
\Prog{cron}. The script should probably update the \Attr{CurMatches}
variable, if you
want to restrict the number of grid jobs that can be submitted at one
time. 

For \Condor{advertise}, specify \Arg{UPDATE\_STARTD\_AD} for
the update command. For example, if your ClassAd is specified in a
file named \File{grid-ad} you would do:

\footnotesize
\begin{verbatim}
    condor_advertise UPDATE_STARTD_AD grid-ad
\end{verbatim}
\normalsize

\Condor{advertise} usually uses UDP to transmit your ClassAd. In
wide-area networks, this may be insufficient. You can use TCP by
specifying the \Opt{-tcp} option. 

%%%%%%%%%%%%%%%%%%%%%%%%%%%%%%%%%%%%%%%%%%%%%%%%%%
\subsubsection{Advanced usage}
%%%%%%%%%%%%%%%%%%%%%%%%%%%%%%%%%%%%%%%%%%%%%%%%%%

What if a job fails to run at a grid site due to an error? It will be
returned to the queue, and Condor will attempt to match it and
re-run it at another site. Condor isn't very clever about avoiding
sites that may be bad, but you can give it some assistance. Let's say
that you want to avoid running at the last grid site you ran at. You
could add this to your job description:

\footnotesize
\begin{verbatim}
match_list_length = 1
Rank              = TARGET.Name != LastMatchName0
\end{verbatim}
\normalsize

This will prefer to run at a grid site that was not just tried, but it
will allow the job to be run there if there is no other option. 

When you specify \Opt{match\_list\_length}, you provide an integer N, and
Condor will keep track of the last N matches. The oldest match will be
LastMatchName0, and next oldest will be LastMatchName1, and so on. (See
the \Condor{submit} manual page for more details.) The Rank expression
allows you to specify a numerical ranking for different matches. When
combined with \Opt{match\_list\_length}, you can prefer to avoid sites that
you have already run at. 

In addition, \Condor{submit} has two options to help you control
grid universe job resubmissions and rematching.  See \Opt{globus\_resubmit} and
\Opt{globus\_rematch} in the \Condor{submit} manual page.
These options are independent of \Opt{match\_list\_length}.

There are some new attributes that will be added to the Job ClassAd,
and may be useful to you when you write your rank, requirements,
globus\_resubmit or globus\_rematch option. Please refer to
Section~\ref{user-man-jobad} and read about the following option:

\begin{itemize}
\item NumJobMatches
\item NumGlobusSubmits
\item NumSystemHolds
\item HoldReason
\item ReleaseReason
\item EnteredCurrentStatus
\item LastMatchTime
\item LastRejMatchTime
\item LastRejMatchReason
\end{itemize}

The following example of a command within the submit description file
releases jobs 5 minutes after being held,
increasing the time between releases by 5 minutes each time.
It will continue to retry up to 4 times per Globus
submission, plus 4.
The plus 4 is necessary in case
the job goes on hold before being submitted to Globus, although
this is unlikely.

\footnotesize
\begin{verbatim}
periodic_release = ( NumSystemHolds <= ((NumGlobusSubmits * 4) + 4) ) \
   && (NumGlobusSubmits < 4) && \
   ( HoldReason != "via condor_hold (by user $ENV(USER))" ) && \
   ((CurrentTime - EnteredCurrentStatus) > ( NumSystemHolds *60*5 ))
\end{verbatim}
\normalsize

The following example forces Globus resubmission after a job has
been held 4 times per Globus submission.

\footnotesize
\begin{verbatim}
globus_resubmit = NumSystemHolds == (NumGlobusSubmits + 1) * 4
\end{verbatim}
\normalsize

If you are concerned about unknown or malicious grid sites reporting
to your \Condor{collector}, you should use Condor's security options,
documented in Section~\ref{sec:Security}.
